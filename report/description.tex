\documentclass{jsarticle}
\usepackage[dvipdfmx]{graphicx}
\usepackage{amsmath}
\usepackage{amsfonts}
\usepackage{bm}
\title{\vspace{-4cm}プログラム説明資料}
\author{廣岡大吾}

\makeatletter
\renewcommand{\thefigure}{\thesection.\arabic{figure}}
\@addtoreset{figure}{section}
\renewcommand{\thetable}{\thesection.\arabic{table}}
\@addtoreset{table}{section}
\renewcommand{\theequation}{\thesection.\arabic{equation}}
\@addtoreset{equation}{section}

\newcommand{\argmax}{\mathop{\rm arg~max}\limits}

\begin{document}
\maketitle
\vspace{-1cm}
\section{作成プログラムの説明資料}

\subsection{課題1:subject1.py}
実験の容易な環境のプログラム.
実行するとコードのテストとして各メソッドを実行した場合の結果が表示される.ここでのエージェントの行動は乱数で生成している
\subsection{課題2:subject2.py}
今回行うCartPoleにおいて,ホストプラグラムと環境情報やエージェントの行動をやり取りするインターフェースのプログラム.
実行すると各メソッドを実行した場合の結果が表示される.
\subsection{課題3:subject3.py}
今回行動を行うエージェントのプログラム.
保持するパラメータの値と受け取った観測状態から次の行動を決定する.
実行すると課題1で作成したEasyEnvを環境として,各メソッドを実行した場合の結果が表示される.
\subsection{課題4:subject4.py}
CrossEntropyMethodによってエージェントのパラメータを学習するプログラム.
長くポールを立て続けられたエージェントほど大きい報酬を得ることを利用してパラメータを学習する.
今回は特に学習の収束判定は設けず100エピソードを行う.
実行すると課題1のEasyEnvと課題3のLinearModelによって学習した場合の結果が出力される.簡単な環境のため特に学習は進まない.
\subsection{課題5:subject5.py}
CartPoleの環境に対してエージェントの学習,及び学習結果の評価を行うプログラム.
CrossEntropyMethodによって学習したパラメータを用いて,さらに100エピソードでテストしパラメータを評価する.
テストの100エピソード中95エピソードでポールを立て続けられれば成功となり,パラメータがテキストファイルで保存される.
実行するとテスト及び評価が行われる.
\subsection{課題6:subject6.py}
課題6において今回用いたプログラムのパラメータや設定を変更した場合の挙動を確認するプログラム.
関数train\_exおよびCEM\_exでは得られる報酬を解析の際に取り回しやすくするための変更を加えた.
クラスCPEnv\_exでは課題4で作成したCartPoleEnvクラスを継承し,本課題で行う観測情報の欠損やノイズを発生させるコードを加えた.
関数polt\_witherrorは実験結果をその分散も含めて図示するものである.
関数exp1, exp2, exp3は実験を行うプログラムである.
実行すると各実験が行われ,結果のグラフが画像として保存される.CartPoleEnvを継承したCPEnv\_exの処理に非常に時間がかかる.

\end{document}
